% DO NOT EDIT - automatically generated from metadata.yaml

\def \codeURL{https://github.com/rescience-c/template}
\def \codeDOI{}
\def \codeSWH{}
\def \dataURL{}
\def \dataDOI{}
\def \editorNAME{Benoît Girard}
\def \editorORCID{}
\def \reviewerINAME{}
\def \reviewerIORCID{}
\def \reviewerIINAME{}
\def \reviewerIIORCID{}
\def \dateRECEIVED{06 April 2023}
\def \dateACCEPTED{}
\def \datePUBLISHED{}
\def \articleTITLE{[Re] Inter-areal Balanced Amplification Enhances Signal Propagation in a Large-Scale Circuit Model of the Primate Cortex}
\def \articleTYPE{Replication}
\def \articleDOMAIN{Computational Neuroscience}
\def \articleBIBLIOGRAPHY{bibliography.bib}
\def \articleYEAR{2019}
\def \reviewURL{https://github.com/ReScience/submissions/issues/72}
\def \articleABSTRACT{Understanding reliable signal transmission repre-sents a notable challenge for cortical systems, whichdisplay a wide range of weights of feedforward andfeedback connections among heterogeneous areas.We re-examine the question of signal transmissionacross the cortex in a network model based onmesoscopic directed and weighted inter-areal con-nectivity data of the macaque cortex. Our findingsreveal that, in contrast to purely feedforward propa-gation models, the presence of long-range excitatoryfeedback projections could compromise stablesignal propagation. Using population rate modelsas well as a spiking network model, we find thateffective signal propagation can be accomplishedby balanced amplification across cortical areas whileensuring dynamical stability. Moreover, the activa-tion of prefrontal cortex in our model requires theinput strength to exceed a threshold, which isconsistent with the ignition model of conscious pro-cessing. These findings demonstrate our model asan anatomically realistic platform for investigationsof global primate cortex dynamics.}
\def \replicationCITE{}
\def \replicationBIB{}
\def \replicationURL{}
\def \replicationDOI{}
\def \contactNAME{Vinicius Lima}
\def \contactEMAIL{vinicius.lima-cordeiro@univ-amu.fr}
\def \articleKEYWORDS{Amygdala}
\def \journalNAME{ReScience C}
\def \journalVOLUME{4}
\def \journalISSUE{1}
\def \articleNUMBER{}
\def \articleDOI{}
\def \authorsFULL{Vinicus Lima, Renan Oliveira Shimoura, Nilton Liuji Kamiji, Demian Battaglia, Antonio Carlos Roque da Silva Filho}
\def \authorsABBRV{V. Lima, R.O. Shimoura, N.L. Kamiji, D. Battaglia, A.C. Roque}
\def \authorsSHORT{VL, ROS, NLK, DM, ACR}
\title{\articleTITLE}
\date{}
\author[1,2,\orcid{0000-0001-7115-9041}]{Vinicius Lima}
\author[3,2,\orcid{0000-0002-6580-5999}]{Renan O. Shimoura}
\author[3\orcid{0000-0001-5006-6612}]{Nilton L. Kamiji}
\author[1, 4, \orcid{0000-0003-2021-7920}]{Demian Battaglia}
\author[3,\orcid{0000-0003-1260-4840}]{Antonio C. Roque}

\affil[1]{Aix-Marseille University, Institute for Systems Neuroscience (INS UMR 1106, Inserm), Marseille, France}

\affil[2]{These authors contributed equally to this work.}


\affil[3]{Department of Physics, School of Philosophy, Sciences and Letters of Ribeirão Preto (FFCLRP), University of São Paulo, Ribeirão Preto, Brazil}
\affil[4]{University of Strasbourg Institute for Advanced Studies (USIAS), Strasbourg, France}
